
% ------------------------------------------------------------------------
% abnTeX2: Modelo de Trabalho Academico (tese de doutorado, dissertacao de
% mestrado e trabalhos monograficos em geral) em conformidade com
% ABNT NBR 14724:2011: Informacao e documentacao - Trabalhos academicos -
% Apresentacao
% ------------------------------------------------------------------------
% ------------------------------------------------------------------------

\documentclass[
  % -- opções da classe memoir --
  12pt,       % tamanho da fonte
  openright,      % capítulos começam em pág ímpar (insere página vazia caso preciso)
  twoside,      % para impressão em verso e anverso. Oposto a oneside
  a4paper,      % tamanho do papel.
  % -- opções da classe abntex2 --
  %chapter=TITLE,   % títulos de capítulos convertidos em letras maiúsculas
  %section=TITLE,   % títulos de seções convertidos em letras maiúsculas
  %subsection=TITLE,  % títulos de subseções convertidos em letras maiúsculas
  %subsubsection=TITLE,% títulos de subsubseções convertidos em letras maiúsculas
  % -- opções do pacote babel --
  english,      % idioma adicional para hifenização
  french,       % idioma adicional para hifenização
  spanish,      % idioma adicional para hifenização
  brazil,       % o último idioma é o principal do documento
  ]{abntex2}


% ---
% PACOTES
% ---

% ---
% Pacotes fundamentais
% ---
\usepackage{cmap}       % Mapear caracteres especiais no PDF
\usepackage{lmodern}      % Usa a fonte Latin Modern
\usepackage[T1]{fontenc}    % Selecao de codigos de fonte.
\usepackage[utf8]{inputenc}   % Codificacao do documento (conversão automática dos acentos)
\usepackage{lastpage}     % Usado pela Ficha catalográfica
\usepackage{indentfirst}    % Indenta o primeiro parágrafo de cada seção.
\usepackage{color}        % Controle das cores
\usepackage{graphicx}     % Inclusão de gráficos
% ---

% ---
% Pacotes adicionais, usados apenas no âmbito do Modelo Canônico do abnteX2
% ---
\usepackage{lipsum}       % para geração de dummy text
% ---

% ---
% Pacotes de citações
% ---
\usepackage[brazilian,hyperpageref]{backref}   % Paginas com as citações na bibl
\usepackage[alf]{abntex2cite} % Citações padrão ABNT

% ---
% CONFIGURAÇÕES DE PACOTES
% ---

% ---
% Configurações do pacote backref
% Usado sem a opção hyperpageref de backref
\renewcommand{\backrefpagesname}{Citado na(s) página(s):~}
% Texto padrão antes do número das páginas
\renewcommand{\backref}{}
% Define os textos da citação
\renewcommand*{\backrefalt}[4]{
  \ifcase #1 %
    Nenhuma citação no texto.%
  \or
    Citado na página #2.%
  \else
    Citado #1 vezes nas páginas #2.%
  \fi}%
% ---


% ---
% Informações de dados para CAPA e FOLHA DE ROSTO
% ---
\titulo{Um framework para geração de testes automatizados para aplicações mobile}
\autor{Gustavo Figueira Olegário}
\local{Brasil}
\data{2017}
\orientador{Ricardo Pereira e Silva}
\coorientador{}
\instituicao{%
  Universidade Federal de Santa Catarina
  \par
  Centro Tecnológico - CTC
  \par
  Departamento de Informática e Estatística
  \par
  Ciências da Computação}
\tipotrabalho{Dissertação (Bacharelado)}
% O preambulo deve conter o tipo do trabalho, o objetivo,
% o nome da instituição e a área de concentração
\preambulo{Trabalho de Conclusão de Curso submetido ao Curso de
Ciências da Computação para a obtenção do Grau de Bacharel em
Ciências da Computação.}
% ---


% ---
% Configurações de aparência do PDF final

% alterando o aspecto da cor azul
\definecolor{blue}{RGB}{41,5,195}

% informações do PDF
\makeatletter
\hypersetup{
      %pagebackref=true,
    pdftitle={\@title},
    pdfauthor={\@author},
      pdfsubject={\imprimirpreambulo},
      pdfcreator={LaTeX with abnTeX2},
    pdfkeywords={abnt}{latex}{abntex}{abntex2}{trabalho acadêmico},
    colorlinks=true,          % false: boxed links; true: colored links
      linkcolor=blue,           % color of internal links
      citecolor=blue,           % color of links to bibliography
      filecolor=magenta,          % color of file links
    urlcolor=blue,
    bookmarksdepth=4
}
\makeatother
% ---

% ---
% Espaçamentos entre linhas e parágrafos
% ---

% O tamanho do parágrafo é dado por:
\setlength{\parindent}{1.3cm}

% Controle do espaçamento entre um parágrafo e outro:
\setlength{\parskip}{0.2cm}  % tente também \onelineskip

% ---
% compila o indice
% ---
\makeindex
% ---

% ----
% Início do documento
% ----
\begin{document}

% Retira espaço extra obsoleto entre as frases.
\frenchspacing

% ----------------------------------------------------------
% ELEMENTOS PRÉ-TEXTUAIS
% ----------------------------------------------------------
% \pretextual

% ---
% Capa
% ---
\imprimircapa
% ---

% ---
% Folha de rosto
% (o * indica que haverá a ficha bibliográfica)
% ---
\imprimirfolhaderosto*
% ---

% ---
% Inserir a ficha bibliografica
% ---

% Isto é um exemplo de Ficha Catalográfica, ou ``Dados internacionais de
% catalogação-na-publicação''. Você pode utilizar este modelo como referência.
% Porém, provavelmente a biblioteca da sua universidade lhe fornecerá um PDF
% com a ficha catalográfica definitiva após a defesa do trabalho. Quando estiver
% com o documento, salve-o como PDF no diretório do seu projeto e substitua todo
% o conteúdo de implementação deste arquivo pelo comando abaixo:
%
% \begin{fichacatalografica}
%     \includepdf{fig_ficha_catalografica.pdf}
% \end{fichacatalografica}
\begin{fichacatalografica}
  \vspace*{\fill}         % Posição vertical
  \hrule              % Linha horizontal
  \begin{center}          % Minipage Centralizado
  \begin{minipage}[c]{12.5cm}   % Largura

  \imprimirautor

  \hspace{0.5cm} \imprimirtitulo  / \imprimirautor. --
  \imprimirlocal, \imprimirdata-

  \hspace{0.5cm} \pageref{LastPage} p. : il. (algumas color.) ; 30 cm.\\

  \hspace{0.5cm} \imprimirorientadorRotulo~\imprimirorientador\\

  \hspace{0.5cm}
  \parbox[t]{\textwidth}{\imprimirtipotrabalho~--~\imprimirinstituicao,
  \imprimirdata.}\\

  \hspace{0.5cm}
    1. Framework.
    2. Testes.
    I. Ricardo Pereira e Silva.
    II. Universidade Federal de Santa Catarina.
    III. Bacharelado em Ciências da Computação.
    IV. Um framework para geração de testes automatizados para aplicações mobile

  \hspace{8.75cm} CDU 02:141:005.7\\

  \end{minipage}
  \end{center}
  \hrule
\end{fichacatalografica}
% ---

% ---
% Inserir errata
% ---
%\begin{errata}
%Elemento opcional da \citeonline[4.2.1.2]{NBR14724:2011}. Exemplo:

%\vspace{\onelineskip}

%FERRIGNO, C. R. A. \textbf{Tratamento de neoplasias ósseas %apendiculares com
%reimplantação de enxerto ósseo autólogo autoclavado associado ao %plasma
%rico em plaquetas}: estudo crítico na cirurgia de preservação de %membro em
%cães. 2011. 128 f. Tese (Livre-Docência) - Faculdade de Medicina %Veterinária e
%Zootecnia, Universidade de São Paulo, São Paulo, 2011.

%\begin{table}[htb]
%\center
%\footnotesize
%\begin{tabular}{|p{1.4cm}|p{1cm}|p{3cm}|p{3cm}|}
%  \hline
%   \textbf{Folha} & \textbf{Linha}  & \textbf{Onde se lê}  & %\textbf{Leia-se}  \\
%    \hline
%   1 & 10 & auto-conclavo & autoconclavo\\
%   \hline
%\end{tabular}
%\end{table}

%\end{errata}
% ---

% ---
% Inserir folha de aprovação
% ---

% Isto é um exemplo de Folha de aprovação, elemento obrigatório da NBR
% 14724/2011 (seção 4.2.1.3). Você pode utilizar este modelo até a aprovação
% do trabalho. Após isso, substitua todo o conteúdo deste arquivo por uma
% imagem da página assinada pela banca com o comando abaixo:
%
% \includepdf{folhadeaprovacao_final.pdf}
%
\begin{folhadeaprovacao}

  \begin{center}
    {\ABNTEXchapterfont\large\imprimirautor}

    \vspace*{\fill}\vspace*{\fill}
    {\ABNTEXchapterfont\bfseries\Large\imprimirtitulo}
    \vspace*{\fill}


    \vspace*{\fill}
   \end{center}

   Este Trabalho de Conclusão de Curso foi julgado aprovado para a
   obtenção do Título de Bacharel em Ciências da Computação, e
   aprovado em sua forma final pelo Curso de Ciências da Computação
   da Universidade Federal de Santa Catarina.

   \assinatura{Dr. Prof. \textbf{\imprimirorientador} \\ Orientador}
   \assinatura{Dr. Prof. \textbf{Professor} \\ Convidado 1}
   \assinatura{Dr. Prof. \textbf{Professor} \\ Convidado 2}
   %\assinatura{\textbf{Professor} \\ Convidado 3}
   %\assinatura{\textbf{Professor} \\ Convidado 4}

   \begin{center}
    \vspace*{0.5cm}
    {\large\imprimirlocal}
    \par
    {\large\imprimirdata}
    \vspace*{1cm}
  \end{center}

\end{folhadeaprovacao}
% ---

% ---
% Dedicatória
% ---
\begin{dedicatoria}
   \vspace*{\fill}
   \centering
   \noindent
   \textit{ Este trabalho é dedicado a todos aqueles que jamais desistem dos seus sonhos.} \vspace*{\fill}
\end{dedicatoria}
% ---

% ---
% Agradecimentos
% ---
\begin{agradecimentos}
Os agradecimentos principais são direcionados aos meus pais, irmãos,
amigos fiéis e à minha namorada, que sempre acreditaram em meu potencial
e que me ajudaram, de alguma forma, a concluir minha graduação.

Agradeço também ao professor Ricardo por seu tempo e dedicação neste trabalho e que desde o início esteve disposto a ajudar na orientação.

Por último, mas não menos importante, agradeço a Deus por esta oportunidade. Pois, sem Sua graça, nada disso existiria.

\end{agradecimentos}
% ---

% ---
% Epígrafe
% ---
\begin{epigrafe}
    \vspace*{\fill}
  \begin{flushright}
    \textit{Não se trata de bater duro. Se trata \\
     de quanto você aguenta apanhar e seguir em \\
     frente. O quanto você é capaz de continuar \\
     tentando. É assim que se consegue vencer.}
  \end{flushright}
\end{epigrafe}
% ---

% ---
% RESUMOS
% ---

% resumo em português
\begin{resumo}
  No escopo da engenharia de software, durante o processo de desenvolvimento
 de aplicações, sabe-se que as fases de desenvolvimento e testes são as que,
 costumeiramente, demandam mais tempo. Desta forma, uma ferramenta capaz
 de gerar testes automatizados para certas aplicações, no mundo de
 dispositivos mobile, pode ser vista como uma forma de acelerar o processo de
 desenvolvimento e permitir que os desenvolvedores se concentrem em
 atividades que demandem mais inteligência do que trabalho manual.

 \vspace{\onelineskip}

 \noindent
 \textbf{Palavras-chaves}: engenharia de software. testes. aplicações mobile.
 framework OO
\end{resumo}

% resumo em inglês
\begin{resumo}[Abstract]
 \begin{otherlanguage*}{english}
   In the software engineering scope, through the development process of
   applications, the stages of development and tests, usually, are the
   ones that require more time. Therefore, a tool capable to generate
   automated tests for certain applications, in the scope of mobile
   gadgets, can be seen as an accelerator during the process of
   tests allowing the developers to focus on tasks that demands
   more intelligence than handwork.

   \vspace{\onelineskip}

   \noindent
   \textbf{Key-words}: software engineering. tests. mobile aplication. framework OO
 \end{otherlanguage*}
\end{resumo}
% ---

% ---
% inserir lista de ilustrações
% ---
\pdfbookmark[0]{\listfigurename}{lof}
\listoffigures*
\cleardoublepage
% ---

% ---
% inserir lista de tabelas
% ---
\pdfbookmark[0]{\listtablename}{lot}
\listoftables*
\cleardoublepage
% ---

% ---
% inserir lista de abreviaturas e siglas
% ---
\begin{siglas}
  %\item[Fig.] Area of the $i^{th}$ component
  %\item[456] Isto é um número
  %\item[123] Isto é outro número
  %\item[lauro cesar] este é o meu nome
\end{siglas}
% ---

% ---
% inserir o sumario
% ---
\pdfbookmark[0]{\contentsname}{toc}
\tableofcontents*
\cleardoublepage
% ---



% ----------------------------------------------------------
% ELEMENTOS TEXTUAIS
% ----------------------------------------------------------
\textual

% ----------------------------------------------------------
% Introdução
% ----------------------------------------------------------
\chapter*[Introdução]{Introdução}
\addcontentsline{toc}{chapter}{Introdução}

Desde o tempo em que o homem começou a programação de computadores, a atividade
de teste de software sempre foi uma atividade vista com descaso e que só era
executada se sobrasse tempo durante o projeto. Em alguns casos ela costumava ser
usada como castigo para programadores que não cumpriam com suas funçṍes.

Com a chegada da crise do software, não demorou muito tempo para que
desenvolvedores percebessem que a atividade de criação e execução de testes era
uma atividade extramemente importante e que também deveria ser incluída durante
o planejamento do projeto.

Nos dias atuais, a fase de teste é uma das mais importantes durante o processo
de desenvolvimento de uma aplicação. Isso se deve à vários fatores, sendo um
deles que os métodos mais adotados pelas empresas no mercado atual são métodos
ágeis. Ou seja, o teste deve ser implementado antes mesmo do programador
desenvolver o módulo o qual lhe foi designado.

Entretanto, devido a essa notória importância que a fase de testes ganhou nos
últimos tempos, ela também tem sido uma das partes que mais tem tomado tempo
em projetos e que poderia ser facilmente automatizada em muitas partes já que
costuma ser uma atividade que demanda mais trabalho manual do que inteligência.
Nesse cenário, a proposta desse trabalho é desenvolver uma ferramenta para
geração de testes automatizados para determinadas aplicações mobile, a fim de
que os desenvolvedores possam se focar em outras tarefas.


% ----------------------------------------------------------
% PARTE - preparação da pesquisa
% ----------------------------------------------------------
\part{Preparação da pesquisa}

% ----------------------------------------------------------
% Capitulo com exemplos de comandos inseridos de arquivo externo
% ----------------------------------------------------------

\include{abntex2-modelo-include-comandos}

% ----------------------------------------------------------
% Parte de revisãod e literatura
% ----------------------------------------------------------
\part{Revisão de Literatura}

% ---
% Capitulo de revisão de literatura
% ---
\chapter{Lorem ipsum dolor sit amet}

% ---
\section{Aliquam vestibulum fringilla lorem}
% ---

\lipsum[1]

\lipsum[2-3]

% ----------------------------------------------------------
% Resultados
% ----------------------------------------------------------
\part{Resultados}

% ---
% primeiro capitulo de Resultados
% ---
\chapter{Lectus lobortis condimentum}

% ---
\section{Vestibulum ante ipsum primis in faucibus orci luctus et ultrices
posuere cubilia Curae}
% ---

\lipsum[21-22]

% ---
% segundo capitulo de Resultados
% ---
\chapter{Nam sed tellus sit amet lectus urna ullamcorper tristique interdum
elementum}

\section{Pellentesque sit amet pede ac sem eleifend consectetuer}

\lipsum[24]

% ---
% Finaliza a parte no bookmark do PDF, para que se inicie o bookmark na raiz
% ---
\bookmarksetup{startatroot}%
% ---

% ---
% Conclusão
% ---
\chapter*[Conclusão]{Conclusão}
\addcontentsline{toc}{chapter}{Conclusão}

\lipsum[31-33]

% ----------------------------------------------------------
% ELEMENTOS PÓS-TEXTUAIS
% ----------------------------------------------------------
\postextual


% ----------------------------------------------------------
% Referências bibliográficas
% ----------------------------------------------------------
\bibliography{abntex2-modelo-references}

% ----------------------------------------------------------
% Glossário
% ----------------------------------------------------------
%
% Consulte o manual da classe abntex2 para orientações sobre o glossário.
%
%\glossary

% ----------------------------------------------------------
% Apêndices
% ----------------------------------------------------------

% ---
% Inicia os apêndices
% ---
\begin{apendicesenv}

% Imprime uma página indicando o início dos apêndices
\partapendices

% ----------------------------------------------------------
\chapter{Quisque libero justo}
% ----------------------------------------------------------

\lipsum[50]

% ----------------------------------------------------------
\chapter{Nullam elementum urna vel imperdiet sodales elit ipsum pharetra ligula
ac pretium ante justo a nulla curabitur tristique arcu eu metus}
% ----------------------------------------------------------
\lipsum[55-57]

\end{apendicesenv}
% ---


% ----------------------------------------------------------
% Anexos
% ----------------------------------------------------------

% ---
% Inicia os anexos
% ---
\begin{anexosenv}

% Imprime uma página indicando o início dos anexos
\partanexos

% ---
\chapter{Morbi ultrices rutrum lorem.}
% ---
\lipsum[30]

% ---
\chapter{Cras non urna sed feugiat cum sociis natoque penatibus et magnis dis
parturient montes nascetur ridiculus mus}
% ---

\lipsum[31]

% ---
\chapter{Fusce facilisis lacinia dui}
% ---

\lipsum[32]

\end{anexosenv}

%---------------------------------------------------------------------
% INDICE REMISSIVO
%---------------------------------------------------------------------

\printindex

\end{document}
