\documentclass[glossy]{beamer}
\useoutertheme{wuerzburg}
\useinnertheme[realshadow,corners=2pt,padding=2pt]{chamfered}
\usecolortheme{shark}

\usepackage{tikz}
\usepackage[utf8]{inputenc}
\newcommand<>{\hover}[1]{\uncover#2{%
    \begin{tikzpicture}[remember picture,overlay]%
        \draw[fill,opacity=0.4] (current page.south west)
        rectangle (current page.north east);
        \node at (current page.center) {#1};
    \end{tikzpicture}}
}

\title{Um framework para geração de testes automatizados para aplicações mobile}
\author{Gustavo Figueira Olegário}
\institute{UFSC}

\begin{document}
    \begin{frame}
        \maketitle
    \end{frame}

    \begin{frame}
        \frametitle{Sumário}
        \begin{itemize}
            \item Frameworks
            \item Teste em nível de usuário
            \item Capuccino
            \item Análise e implementação
            \item Resultados obtidos
        \end{itemize}
    \end{frame}
    \begin{frame}
    \frametitle{Frameworks}
        \begin{itemize}
            \item Nos primeiros projetos de software, o reuso era praticamente nulo
            \item Usuário deve redefinir algumas classes
            \item Permite reuso de componentes de problemas que compartilham mesmo domínio do problema
            \item Desvantagem: equipe deve saber como usar o framework e fica presa à arquitetura definida
                    pela ferramenta.
        \end{itemize}
    \end{frame}
\end{document}
